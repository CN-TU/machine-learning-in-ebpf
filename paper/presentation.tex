
%
% Beispieldokument für TU beamer theme
%
% v1.0: 12.10.2014
% 
\documentclass[xcolor={dvipsnames}]{beamer}
\usetheme{TU}

\setbeamertemplate{caption}[numbered]
\usepackage{hyperref}
\usepackage{booktabs}
\usepackage{tabularx}
\usepackage{subfig}
% Macro to make entire row bold from https://tex.stackexchange.com/questions/309833/format-whole-row-of-table-as-bold
\newcommand\setrow[1]{\gdef\rowmac{#1}#1\ignorespaces}
\newcommand\clearrow{\global\let\rowmac\relax}
\clearrow

\title{A flow-based IDS using Machine Learning in eBPF}

\author[M. Bachl et al.]{%
	\underline{Maximilian Bachl}\email{maximilian.bachl@tuwien.ac.at} \and Joachim Fabini \and Tanja Zseby
}

\institute{%
	Technische Universität Wien, Vienna, Austria
}

%Kann angepasst werden, wie es beliebt. Entweder das Datum der Präsentation oder das Datum der aktuellen Präsentations-Version.
%\date[\the\day.\the\month.\the\year]{\today}
\date[February 16, 2021]{February 16, 2021}

\begin{document}

\maketitle

\section{Introduction}

\begin{frame}{Packet sending/receiving in Linux}
\centering
\includegraphics[width=0.8\columnwidth]{{"figures/Linux networking"}.png}

\tiny Illustration from \url{https://conferences.sigcomm.org/sigcomm/2019/files/slides/paper_2_4.pptx}
\end{frame}

\begin{frame}{Packet sending/receiving in Linux: Insight}
\begin{block}{Insight:}
Passing a packet from the network interface to the application is pretty complicated and computationally expensive.
\end{block}
\end{frame}

\begin{frame}{Solution}
\begin{itemize}
\item Filter unwanted packets immediately after receiving them
\item This saves a lot of computational resources 
\end{itemize}
\end{frame}

\begin{frame}{BPF (Berkeley Packet Filter)}
\begin{itemize}
\item Programs tell operating system (OS) to filter packets for them
\item Programs give OS a mini program with instructions on what to filter
\end{itemize}
\end{frame}

\begin{frame}{Example: tcpdump}
\begin{itemize}
\item tcpdump records relevant packets
\item Example: \texttt{tcpdump icmp and dst 51.254.96.253}
\item tcpdump gives the filter as BPF program to the OS
\item This makes filtering more efficient
\end{itemize}
\end{frame}

\begin{frame}{Use BPF everywhere in the OS kernel!}
\begin{block}{Insight:}
BPF could be used for modifying the OS dynamically! 
\end{block}
\end{frame}

\begin{frame}{eBPF}
\begin{itemize}
\item Make BPF more powerful
\item Allow more locations in the OS to be customized with eBPF
\item Any C program can be compiled to eBPF
\item eBPF programs are verified before being loaded into the OS kernel:
\begin{itemize}
\item eBPF programs cannot crash
\item eBPF programs cannot run infinitely
\end{itemize}
\end{itemize}
\end{frame}

\begin{frame}{eBPF -- Disadvantages}
\begin{itemize}
\item eBPF cannot be turing complete
\item If eBPF were turing complete you couldn't verify that programs can't crash
\item Specifically: eBPF cannot use for-loops of arbitrary length and while-loops
\item eBPF uses special data structures to make sure that no out-of-bounds accesses can happen
\end{itemize}
\end{frame}

\begin{frame}{eBPF for machine learning}
\begin{itemize}
\item Training/deploying neural networks doesn't require loops of arbitrary length
\item Training/deploying decision trees/random forests doesn't require loops either
\end{itemize}
\pause
\begin{block}{Insight:}
Can use machine learning with eBPF in the kernel
\end{block}
\end{frame}

\section{Concept}
\begin{frame}{eBPF based IDS}
\begin{itemize}
\item Use 5 tuple to identify flows
\item Create hashmap to store flow info (hashmaps are built into eBPF)
\item Features: the source and destination port, the protocol identifier (UDP, TCP, ICMP etc.), the packet length, the time since the last packet of the flow and the direction of the packet
\item Also use the mean of the feature values as well as mean absolute deviation
\end{itemize}
\end{frame}

\begin{frame}{eBPF based IDS -- limitations}
\begin{itemize}
\item eBPF doesn't support floating point numbers
\item Alternative to floating point is fixed point arithmetics. Use 16 bits for the part behind the dot and 48 for the part before the dot
\item Also, it doesn't support square roots etc. Thus can't use standard deviation but must use mean absolute deviation
\end{itemize}
\end{frame}

\begin{frame}{eBPF based IDS -- implementation}
\begin{itemize}
\item Train decision tree in sklearn using CIC-IDS-2017
\item Limit to depth 10 and a maximum of 1000 leaves
\item Achieves accuracy of 99\%
\end{itemize}
\end{frame}

\section{Evaluation}
\begin{frame}{Setup}
\begin{itemize}
\item Everything on one machine using Linux network namespaces
\item Two hosts connected via a switch
\item iPerf for bulk transfer
\item eBPF-based IDS listens on raw socket for all packets on the server 
\end{itemize}
\end{frame}

\begin{frame}{Evaluation method}
\begin{itemize}
\item Run same C program once as eBPF and once as userspace program
\item eBPF program is identical to userspace program except for the data structures
\item Run each for 10 seconds
\item Compare which solution could process more packets
\end{itemize}
\end{frame}

\begin{frame}{Results}
\begin{table}[h]
\caption{The maximum number of packets which each implementation can process.} \label{tab:comparison}
\centering
\begin{tabular}{rrr} \toprule
& Userspace & eBPF \\ \midrule
packets/s & 123\,800 & 153\,290 \\
\bottomrule
\end{tabular}
\end{table}
\end{frame}

\begin{frame}{Interpretation}
\begin{itemize}
\item eBPF is 24\% faster than the userspace program
\end{itemize}
\end{frame}

\begin{frame}{Discussion}
\begin{itemize}
\item It seems obvious that eBPF is faster
\item But it uses different data structures
\item eBPF can't use normal arrays because they can have out-of-bounds accesses
\item eBPF's data structures have an overhead compared to native C ones
\end{itemize}
\end{frame}

\begin{frame}{Future work}
\begin{itemize}
\item What's the point at which the overhead from eBPF's data structures becomes larger than the benefit?
\item Can random forests with many trees and thousands of leaves still be efficient?
\item Can deep neural networks still be efficient when using eBPF's data structures?
\end{itemize}
\end{frame}


\makelastslide

\end{document}