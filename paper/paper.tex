%%
%% This is file `sample-sigconf.tex',
%% generated with the docstrip utility.
%%
%% The original source files were:
%%
%% samples.dtx  (with options: `sigconf')
%%
%% IMPORTANT NOTICE:
%%
%% For the copyright see the source file.
%%
%% Any modified versions of this file must be renamed
%% with new filenames distinct from sample-sigconf.tex.
%%
%% For distribution of the original source see the terms
%% for copying and modification in the file samples.dtx.
%%
%% This generated file may be distributed as long as the
%% original source files, as listed above, are part of the
%% same distribution. (The sources need not necessarily be
%% in the same archive or directory.)
%%
%% The first command in your LaTeX source must be the \documentclass command.
\documentclass[conference]{IEEEtran}

\usepackage[utf8]{inputenc}
\usepackage{multirow}
\usepackage[inline]{enumitem}
\usepackage{xcolor}
\usepackage{booktabs}
\usepackage[pdfauthor={Maximilian Bachl, Joachim Fabini, Tanja Zseby}]{hyperref}
\usepackage{amsmath}
\usepackage{amssymb}
\usepackage{graphicx}
\usepackage[numbers]{natbib}
\usepackage{subfig}
\usepackage{tikz}

\usepackage[nomain, toc, acronym]{glossaries}
\glsdisablehyper

\newcommand\note[2]{{\color{#1}#2}}
\newcommand\todo[1]{{\note{red}{TODO: #1}}}

%%
%% end of the preamble, start of the body of the document source.
\begin{document}

%%
%% The "title" command has an optional parameter,
%% allowing the author to define a "short title" to be used in page headers.
\title{A flow-based IDS using Machine Learning in eBPF}

\author{\IEEEauthorblockN{Maximilian Bachl, Joachim Fabini, Tanja Zseby}
\IEEEauthorblockA{Technische Universität Wien\\
firstname.lastname@tuwien.ac.at}}

%\date{\today}

% \IEEEoverridecommandlockouts
% \IEEEpubid{\begin{minipage}[t]{\textwidth}\ \\[10pt]
%         \centering\normalsize{xxx-x-xxxx-xxxx-x/xx/\$31.00 \copyright 2018 IEEE}
% \end{minipage}}

% \renewcommand*{\bibfont}{\footnotesize}

%\newcommand\copyrighttext{%
%  \footnotesize \textcopyright 2020 IEEE. Personal use of this material is permitted.
%  Permission from IEEE must be obtained for all other uses, in any current or future
%  media, including reprinting/republishing this material for advertising or promotional
%  purposes, creating new collective works, for resale or redistribution to servers or
%  lists, or reuse of any copyrighted component of this work in other works.}
%\newcommand\copyrightnotice{%
%\begin{tikzpicture}[remember picture,overlay]
%\node[anchor=south,yshift=10pt] at (current page.south) {\fbox{\parbox{\dimexpr\textwidth-\fboxsep-\fboxrule\relax}{\copyrighttext}}};
%\end{tikzpicture}%
%}

\maketitle%
%\copyrightnotice

% \thispagestyle{plain}
% \pagestyle{plain}

\newacronym{ml}{ML}{Machine Learning}
\newacronym{dl}{DL}{Deep Learning}
\newacronym{ids}{IDS}{Intrusion Detection System}
\newacronym{rnn}{RNN}{Recurrent Neural Network}
\newacronym{dos}{DoS}{Denial-of-Service}
\newacronym{iat}{IAT}{Interarrival time}

\begin{abstract}
eBPF is a new technology which allows dynamically loading pieces of code into the Linux kernel. For example, it can greatly speed up networking since it enables the kernel to process certain packets without the involvement of a user space program. So far eBPF has been used for simple packet filtering applications such as firewalls or Denial of Service protection. We show that it is possible to develop a flow based network intrusion detection system based on machine learning entirely in eBPF. Our solution uses a decision tree and decides for each packet whether it is malicious or not, considering the entire previous context of the network flow. We achieve a performance increase of over 30\% compared to the same solution implemented as a user space program. 
\end{abstract}

%%
%% The abstract is a short summary of the work to be presented in the
%% article.
%\begin{abstract}
%  A clear and well-documented \LaTeX\ document is presented as an
%  article formatted for publication by ACM in a conference proceedings
%  or journal publication. Based on the ``acmart'' document class, this
%  article presents and explains many of the common variations, as well
%  as many of the formatting elements an author may use in the
%  preparation of the documentation of their work.
%\end{abstract}

\maketitle

\section{Introduction}


%\section*{Acknowledgements}
%The Titan Xp used for this research was donated by the NVIDIA Corporation.

\renewcommand*{\bibfont}{\small}
\bibliographystyle{ieeetr}
\bibliography{bibliography}


\end{document}
\endinput
